% -*- Mode:TeX -*-

%% The documentclass options along with the pagestyle can be used to generate
%% a technical report, a draft copy, or a regular thesis.  You may need to
%% re-specify the pagestyle after you \include  cover.tex.  For more
%% information, see the first few lines of mitthesis.cls. 

%\documentclass[12pt,vi,twoside]{mitthesis}
%%
%%  If you want your thesis copyright to you instead of MIT, use the
%%  ``vi'' option, as above.
%%
%\documentclass[12pt,twoside,leftblank]{mitthesis}
%%
%% If you want blank pages before new chapters to be labelled ``This
%% Page Intentionally Left Blank'', use the ``leftblank'' option, as
%% above. 

\documentclass[12pt,twoside]{mitthesis}
\usepackage{lgrind} 		  % 2e style for including codes

%% Additional packages
% \usepackage[utf8]{inputenc} % allow utf-8 input
% \usepackage[T1]{fontenc}    % use 8-bit T1 fonts
% \usepackage{hyperref}       % hyperlinks
% \usepackage{url}            % simple URL typesetting
\usepackage{booktabs}       % professional-quality tables
% \usepackage{amsfonts}       % blackboard math symbols
% \usepackage{nicefrac}       % compact symbols for 1/2, etc.
% \usepackage{microtype}      % microtypography
\usepackage{graphicx}
\usepackage{subfig}
\usepackage{pdflscape}
\usepackage{multirow}
\usepackage{afterpage}
% \usepackage[table,x11names]{xcolor}

\pagestyle{plain}

%% This bit allows you to either specify only the files which you wish to
%% process, or `all' to process all files which you \include.
%% Krishna Sethuraman (1990).
%
%\typein [\files]{Enter file names to process, (chap1,chap2 ...), or `all' to process all files:}
%\def\all{all}
%\ifx\files\all \typeout{Including all files.} \else \typeout{Including only \files.} \includeonly{\files} \fi

%% This is an example first chapter.  You should put chapter/appendix that you
%% write into a separate file, and add a line \include{yourfilename} to
%% main.tex, where `yourfilename.tex' is the name of the chapter/appendix file.
%% You can process specific files by typing their names in at the 
%% \files=
%% prompt when you run the file main.tex through LaTeX.

\begin{document}

% -*-latex-*-
% 
% For questions, comments, concerns or complaints:
% thesis@mit.edu
% 
% NOTE:
% These templates make an effort to conform to the MIT Thesis specifications,
% however the specifications can change.  We recommend that you verify the
% layout of your title page with your thesis advisor and/or the MIT 
% Libraries before printing your final copy.
\title{Sloop: Pattern Retrieval System for Animal Biometrics}

\author{Navi Tansaraviput}
% If you wish to list your previous degrees on the cover page, use the 
% previous degrees command:
%       \prevdegrees{A.A., Harvard University (1985)}
% You can use the \\ command to list multiple previous degrees
%       \prevdegrees{B.S., University of California (1978) \\
%                    S.M., Massachusetts Institute of Technology (1981)}
\department{Department of Electrical Engineering and Computer Science}

% If the thesis is for two degrees simultaneously, list them both
% separated by \and like this:
% \degree{Doctor of Philosophy \and Master of Science}
\degree{Master of Engineering in Computer Science and Engineering}

% As of the 2007-08 academic year, valid degree months are September, 
% February, or June.  The default is June.
\degreemonth{September}
\degreeyear{2016}
\thesisdate{September 12, 2016}

%% By default, the thesis will be copyrighted to MIT.  If you need to copyright
%% the thesis to yourself, just specify the `vi' documentclass option.  If for
%% some reason you want to exactly specify the copyright notice text, you can
%% use the \copyrightnoticetext command.  
%\copyrightnoticetext{\copyright IBM, 1990.  Do not open till Xmas.}

% If there is more than one supervisor, use the \supervisor command
% once for each.
\supervisor{Srinivas Ravela}{Principal Research Scientist}

% This is the department committee chairman, not the thesis committee
% chairman.  You should replace this with your Department's Committee
% Chairman.
\chairman{Leslie A. Kolodziejski}{Chairman, Department Committee on Graduate Theses}

% Make the titlepage based on the above information.  If you need
% something special and can't use the standard form, you can specify
% the exact text of the titlepage yourself.  Put it in a titlepage
% environment and leave blank lines where you want vertical space.
% The spaces will be adjusted to fill the entire page.  The dotted
% lines for the signatures are made with the \signature command.
\maketitle

% The abstractpage environment sets up everything on the page except
% the text itself.  The title and other header material are put at the
% top of the page, and the supervisors are listed at the bottom.  A
% new page is begun both before and after.  Of course, an abstract may
% be more than one page itself.  If you need more control over the
% format of the page, you can use the abstract environment, which puts
% the word "Abstract" at the beginning and single spaces its text.

%% You can either \input (*not* \include) your abstract file, or you can put
%% the text of the abstract directly between the \begin{abstractpage} and
%% \end{abstractpage} commands.

% First copy: start a new page, and save the page number.
\cleardoublepage%
% Uncomment the next line if you do NOT want a page number on your
% abstract and acknowledgments pages.
% \pagestyle{empty}
\setcounter{savepage}{\thepage}
\begin{abstractpage}
% $Log: abstract.tex,v $
% Revision 1.1  93/05/14  14:56:25  starflt
% Initial revision
%
% Revision 1.1  90/05/04  10:41:01  lwvanels
% Initial revision
%
%
%% The text of your abstract and nothing else (other than comments) goes here.
%% It will be single-spaced and the rest of the text that is supposed to go on
%% the abstract page will be generated by the abstractpage environment.  This
%% file should be \input (not \include 'd) from cover.tex.


% In this thesis, I designed and implemented a compiler which performs
% optimizations that reduce the number of low-level floating point operations
% necessary for a specific task; this involves the optimization of chains of
% floating point operations as well as the implementation of a ``fixed'' point
% data type that allows some floating point operations to simulated with
% integer arithmetic.  The source language of the compiler is a subset of C,
% and the destination language is assembly languagesadsdasdsd for a
% micro-floating point CPU.  An instruction-level simulator of the CPU was
% written to allow testing of the code.  A series of test pieces of codes was
% compiled, both with and without optimization, to determine how effective
% these optimizations were.

The ability to identify individual animals is crucial for non-invasive
ecological monitoring and conservation planning. This project proposed two
improvements to the recognition process and ranked retrival of Sloop, the first
image retrieval engine that couples automated pattern recognition with
crowd-sourced relevance feedback for individual animal identification.

With a crowd-sourced relevance feedback simulator, we report a number of
studies corroborating the acceration of precision and recall of the retreival
results after various rounds of relevance feedback and the effects of the error
propagation.

Then, we describe Sloop MTurk, the crowdsourced relevance feedback integration
of Sloop.

In the later part, we propose a new architecture for animal pattern
recognition, which could possibly reduce the system necessity for human
involvement in feature extraction using transfer learning. Then, we experiment
with a variety of binary classifiers in order to identify the algorithm that
accomplish good performance on our data. Our results reveal that Convolutional
network with linear support vector machine with radial basis kernel function
(SVM-RBF) achieves a very robust performance on Otago and Grand data.



\end{abstractpage}

% Additional copy: start a new page, and reset the page number.  This way,
% the second copy of the abstract is not counted as separate pages.
% Uncomment the next 6 lines if you need two copies of the abstract
% page.
% \setcounter{page}{\thesavepage}
% \begin{abstractpage}
% % $Log: abstract.tex,v $
% Revision 1.1  93/05/14  14:56:25  starflt
% Initial revision
%
% Revision 1.1  90/05/04  10:41:01  lwvanels
% Initial revision
%
%
%% The text of your abstract and nothing else (other than comments) goes here.
%% It will be single-spaced and the rest of the text that is supposed to go on
%% the abstract page will be generated by the abstractpage environment.  This
%% file should be \input (not \include 'd) from cover.tex.


% In this thesis, I designed and implemented a compiler which performs
% optimizations that reduce the number of low-level floating point operations
% necessary for a specific task; this involves the optimization of chains of
% floating point operations as well as the implementation of a ``fixed'' point
% data type that allows some floating point operations to simulated with
% integer arithmetic.  The source language of the compiler is a subset of C,
% and the destination language is assembly languagesadsdasdsd for a
% micro-floating point CPU.  An instruction-level simulator of the CPU was
% written to allow testing of the code.  A series of test pieces of codes was
% compiled, both with and without optimization, to determine how effective
% these optimizations were.

The ability to identify individual animals is crucial for non-invasive
ecological monitoring and conservation planning. This project proposed two
improvements to the recognition process and ranked retrival of Sloop, the first
image retrieval engine that couples automated pattern recognition with
crowd-sourced relevance feedback for individual animal identification.

With a crowd-sourced relevance feedback simulator, we report a number of
studies corroborating the acceration of precision and recall of the retreival
results after various rounds of relevance feedback and the effects of the error
propagation.

Then, we describe Sloop MTurk, the crowdsourced relevance feedback integration
of Sloop.

In the later part, we propose a new architecture for animal pattern
recognition, which could possibly reduce the system necessity for human
involvement in feature extraction using transfer learning. Then, we experiment
with a variety of binary classifiers in order to identify the algorithm that
accomplish good performance on our data. Our results reveal that Convolutional
network with linear support vector machine with radial basis kernel function
(SVM-RBF) achieves a very robust performance on Otago and Grand data.



% \end{abstractpage}

\cleardoublepage%

\section*{Acknowledgments}

This thesis would not have been possible without the support and help from many
individuals to whom I am sincerely appreciated.

First, I would like to express my deepest gratitude to my advisor, Dr.\ Srinivas
Ravela, for his expert guidance, profound understanding, and encouragement
throughout my study and research. Without his counsel and great patience, this
work would truly not have been possible.

I would also like to extend my greatest appreciation to Randy Westlund for
providing indispensable advice, insights, and invaluable guidance in every
aspect of my research, not to mention how much his engineering passion and
enthusiasm have inspired me. His willingness to generously spend time
discussing and his unceasing help were indeed imperative to the completion of this
thesis.

Additionally, I would like to take this opportunity to thank my family for
their unequivocal support for which my mere expression of thanks likewise does
not suffice.

This research is funded by New Zealand Department of Conservation (DOC).

%%%%%%%%%%%%%%%%%%%%%%%%%%%%%%%%%%%%%%%%%%%%%%%%%%%%%%%%%%%%%%%%%%%%%%
% -*-latex-*-

% Some departments (e.g. 5) require an additional signature page.  See
% signature.tex for more information and uncomment the following line if
% applicable.
% \include{signature}
\pagestyle{plain}
\include{contents}
% Introduction (context interms of content of the project)
% * Significance (who will benefit? contribution of the study)
% * Statement of the problem (problem must be reflected to the title)
% * Conceptual framework (problems in relation to relevant literatures,
% summerize the major variables -- independent variables/cause,
% dependent/effect, other infkuencial vars)
%   - Existing research and its relevancy
%   - Key idea of my appproach
%   - discuss variables related to the problem
%   - Conceptualize relationship between variables
% * Scope and delimitation
% (* list out technical terms)

\chapter{Introduction}

Unbiased information about animal population ecology gives biologists vital
information about the effects of different physical or biological factors on
the distribution and abandance of animal species, which plays a key role in the
development of effective conservation strategies for rare and endangered
species. Typical approaches biologists take to estimate the precise size of a
population of species often involves physically marking the animals. These
approaches are not only expensive in terms of time and cost, but they are also
unhealthy for the animals.

As an alternative, ability to identify an individual animal by recognizing its
photographs allows researchers to monitor the species' diversity, and dispersal
in a non-invasive. Researchers can track the movements and observe the genetic
variation of a species by comparing each member's images with the all the
existing images collected from different time and locations. However, the
arduous task of comparing over a thousand images of every individual animal and
their potential matches makes manual reviews impracticable for large
collections. To alliviate the problem, we need to automate the recognition
process using computer-based image recognition techniques.

The considerable progress we have seen in computer vision is largely due to
local descriptor-based algorithms, such as SIFT\cite{lowe04}, and
SURF\cite{surf08}, etc. The field of computer vision covers a wide range of
topics. For the purpose of animal image biometric, this thesis will deal mostly
with the image classification and recognition problems. Despite the stellar
performance of these algorithms in the past decades, there exists a substancial
gap between human-level performance and theirs. In this work, we present two
improvements, whose results have shown to bridge the performance gap between
human and these machine-based algorithms, to \emph{Sloop}, the first pattern
retrieval engine for animal biometric incorperating crowd-sourced relevance
feedback.

In the last few years, deep convolutional neural networks\cite{lecun95, kriz12}
have outperformed SIFTs and other descriptor-based techniques by a large margin
in both object recognition and classification tasks\cite{kriz12, fisher14,
ILSVRC15}. In fact, the architecture has demonstrated recognition accuracy
comparable to humans in several visual recognition tasks, such as recognizing
faces\cite{deepface14}, and handwritten digits\cite{mnist13}. Motivated by the
preceding achievements, we integrate a pre-trained convolutional neural network
into a new python version of Sloop, \emph{SlooPy}, as a seperate image
processing workflow.

Despite the scalability, and advantages of computational speed in automatic
pattern recognition, some classification errors could occur and rapidly
propagate. Some degree of human involvement may benefit the identification
process. Not only can user input resolve the errors, but it can also be used to
train machine learning model. The model can incrementally learn from the
accumulation of user input data in our retrieval system, which create a
positive feedback loop where the model learns its mistakes from the previous
iterations and re-evaluate its strategy based on the gold standard responses
from human given at the previous iteration. However, this is out of the scope
of our work. In this project, we focus on the problem of how we can maximize
the the information gain from a given amount of user input.

This project involves design and implementation of two additional features of
Sloop, an existing pattern retrieval engine for individual animal
identification. The features include: \begin{enumerate} \item Relevance
Feedback Integration \item Convolutional Neural Network Integration
\end{enumerate}

\section{Problem Statements}

\subsection{Relevance Feedback} % (fold) \label{sub:relevance_feedback} One
problem of interest is assigning tasks to workers with the goal of maximizing
the quality of completed tasks at a low prices or subject to budget
constraints.
% subsection relevance_feedback (end)

\subsection{Image Recognition} % (fold) \label{sub:image_recognition} The
emergence of convolutional neural network pushes forward the frontiers of all
domains of computer vision \cite{lecun95}. Recent studies shows that
convolutional neural network architecture clearly dominates the handcrafted
features, and traditional orientation-based local descriptors, such as
SIFT\cite{lowe04}, and SURF\cite{surf08}, etc. in classification
tasks\cite{fisher14,kriz12,prelu15,ILSVRC15}.

The second part of the thesis presents a new architecture whose goal is to
improve the identification accuracy as well as curtail or eliminate human
involvement. We compare the recognition ability of the two algorithms: SIFT and
convolutional neural network. 
% subsection image_recognition (end)

\section{Challenges}

The animal identification task involves two major challenges: \emph{image
feature extraction problem}, and \emph{pattern recognition problem}. 

Image feature extraction problem involves locating the animal in the image, and
extract the features required for the matching. The choice of feature selection
varies from species to species. The extracted feature object of an image is then
passed into the pattern recognition algorithm to find the matches existed in the
system. Current Sloop locates the animal and necessary features of an image by
having the user click on the key points of the animal, and then calculates the
feature vector using SIFT. It performs the matching by scoring the similarity of
the SIFT object. 

Traditional approaches in machine learning generally require training samples to
be available for all the categories. Moreover, such approaches are designed to
handle only the dataset with finite or limited, preferably small, number of
categories. Nevertheless, our application requires ability to recognize high
dimensional input, whose categories are not known in advance. In addition, while
the number of categories can be very large, the number of examples per category
can be very small.

% Introduction (context interms of content of the project)
% * Significance (who will benefit? contribution of the study)
% * Sattement of the problem (problem must be reflacted to the title)
% * Conceptual framework (problems in relation to relevant literatures, summerize the major variables -- independent variables/cause, dependent/effect, other infkuencial vars)
%   - Existing research and its relevancy
%   - Key idea of my appproach
%   - discuss variables related to the problem
%   - Conceptualize relationship between variables
% * Scope and delimitation
% (* list out technical terms)
\graphicspath{{./images/chap2/}}
% Related Liturature and Studies
% * Organzied to cover specific problem
% * how it helps the current study/how it relates
\chapter{Related Work} % (fold)
\label{cha:related_work}

\section{Sloop}

\subsection{Information Retrieval System}

The field of information retrieval emerges from the attempt to provide
information access. From the academic perspective, information retrieval (IR)
is a principled approach of finding desired materials of an \emph{unstructured}
nature from within large collections. The fact that it allows more flexible
query operations makes IR a dominant form of information access in practice,
compare to database-style searching. Additionally, IR supports ranked
retrieval, where it outputs the best answers given a query.

The preceding properties make information retrieval an obvious solution for
animal identification task. Tracking population and dispersal of a species from
images requires manually identifying all the individuals animals in the images.
Traditionally, this involves an arduous work of comparing thousands of images.
However, with Sloop, a pattern retrieval engine that can preprocess the images
and output an initial possible ranking, the time required to spend on going
through all the image pairs one by one could be reduced by an order of
magnitude.

\subsection{Sloop Architecture} Sloop is a distributed image retrieval system.
The system is divided into two major components:

\begin{figure}[h]
  \centering
  \includegraphics[width=\textwidth]{sloop/system}
  \caption{Score Distribution}
  \label{fig:overview}
\end{figure}

\begin{enumerate}
	\item Data Exchange and Interaction (DEI)
    DEI is a web application that provides the user inferface that allows
    biologists to upload data onto sloop databases.
	\item Image Processing Engine (IPE)
    IPE processes the data stored in the databases and generate descriptors
    that represent identities of the images.
\end{enumerate}


Sloop identifies each animal on individual base. This provides the similarity
ranking of the images, within the same species, in the database relative to the
animal in a given image A. Effectiveness of such identification system heavily
depends on the choice of features on which the machine learning algorithms are
applied.

Current Sloop uses Scale Invariant Feature Transform (SIFT) \cite{lowe04} to
perform such images ranking. Given an image and the four fiducial key points
annotated by the biologists, Sloop transforms the images into SIFT object and
compare it among the SIFTs of the existing images stored in the database, using
Euclidean distance. After the ranking is generated, the biologists then look at
the top matches and confirm whether the given pairs of images are matches or
non-matches. 

While SIFT is still interesting for tasks that speed and simplicity are of
major concerns, it requires tremendous amount of manual effort and training.
Recent studies and competition results have proved that features learned via
convolutional neural networks (CNN) outperform previous descriptors, includign
SIFT, on classification tasks by a wide margin\cite{fisher14}. Therefore,
replacing SIFT with convolutional neural networks seem to be a viable
improvement to the system.

\section{Relevance Feedback}

Relvance Feedback is a technique that involves users in the retrival process.
Specifically, given a query that returns a set of initial results, the system
takes user feedback on the initial results to improve the results returned from
the later iterations when given the same or related query
\cite{manning2008introduction}.

\subsection{Crowdsourcing}

Crowdsourcing market is a platform takes advantage of collective intelligence
of the online community. Crowdsourcing has gained popularity as an inexpensive
and efficient method to accomplish certain tasks that are usually difficult for
machines alone to complete.

In a crowdsourcing market, there are three parties involved:
\begin{enumerate}
	\item Workers
	\item Requesters
	\item Crowdsourcing platform
\end{enumerate}

\emph{Requesters} submit tasks with the amount of \emph{reward} that they are
willing to pay \emph{workers} upon the completion of the tasks. Some workers
maybe better than others at certain tasks. In other words, some tasks maybe
more difficult than others for some workers. The platform provides the
environment for the worker and requesters to interact. All parties gain more
information about one another and the tasks, and make repeated decision
overtime.

\section{Image Recognition}

\subsection{Convolutional Neural Network}

Convolutional Neural Network is a special kind of multi-layer neural networks,
whose task is specialized to capture and encode visual patterns directly from
raw pixel images \cite{lecun95}. 

\subsection{Face Verification and Siamese Network}

The problem of finding matching image pairs from the database given an input
image, can be reduced to following \emph{decision problem}:
\begin{quote}
\centering
Given two images A and B, is the animal in image A the same individual as the
animal in image B?
\end{quote}

Our problem is somewhat similar to the face verification problem, which
involves accepting or rejecting an identity claim based on the image of a human
face. There are two major differences between animal identification and face
verification. First, the animal pattern has finer-grained details and some can
be very subtle that they fuse into the background. Second, the pattern of each
individual does not share same overall structure as human face does. This is
similar to identifying an identical twin by their blemishes except for, in this
case, we have very high-order of multiple births.

Most current face verification methods use hand-crafted features, which are
often combined to improve the validation performance. In \cite{chopra05},
Chopra presented a new framework, \emph{Siamese Network}, as a solution to this
problem. He proposed a training method that is used to learn a similarity
metric from the data with the contrastive loss function, comprised of the sum
of the \emph{magnitude of the difference between the features vectors} of the
incorrect guesses. This loss function encourages matching pairs to be close
together in feature space while pushing non-matching pairs apart. However, it
may not be the best option in our case because the difference in each feature
contributes equal weight to the loss, whereas in our model, each feature may
require different weights in the loss function that is not proportional to the
weight learned in the convolutional network.

% chapter related_work (end)

% Related Liturature and Studies
% * Organzied to cover specific problem
% * how it helps the current study/how it relates
\graphicspath{{./images/chap3/}}
% Relevance Feedback
% Methodology
% * Design and Architecture
% * Population of interest and sampling subject used in the study
% * Instrument and what it measures (metrices)
% * qualifications of informants if used in the study
% * Validation
% * Data gathering procedure (experiments)
\chapter{Datasets} % (fold)
\label{cha:datasets}

All the experiments in this study are conducted on a database two species of skinks: \emph{Grand} and \emph{Otago} of 3687 images in total created by biologists at New Zealand Department of Conservation. 

\begin{figure}[h]
  \centering
  \subfloat{\includegraphics[width=4cm]{dataset/general/grand_L3}}
  \subfloat{\includegraphics[width=4cm]{dataset/general/grand_L1}}
  \subfloat{\includegraphics[width=4cm]{dataset/general/grand_L2}}\\

  \subfloat{\includegraphics[width=4cm]{dataset/general/otago_R2}}
  \subfloat{\includegraphics[width=4cm]{dataset/general/otago_R1}}
  \subfloat{\includegraphics[width=4cm]{dataset/general/otago_R3}}\\
  \captionsetup{justification=centering}
  \caption{Top: Left view images from Grand dataset.\\Bottom: Right view images from the Otago dataset}
\end{figure}

The Grand database contains 1871 \texttt{RGB} images of 206 individuals with variations in sizes, lighting, and background, while the Otago database contains 1816 \texttt{RGB} images of 221 individuals, also with such variations. The images are closely cropped to include only the anterior end of the subjects; therefore, the size varies from 1056 $\times$ 564 to 2437 $\times$ 1215 pixels

\begin{table}[t]
\captionsetup{justification=centering}
  \caption{Summary of each database}
  \label{database-table}
  \centering
  \begin{tabular}{llllllll}
    \toprule
    & \multicolumn{3}{c}{Number of Images} & & & \multicolumn{2}{c}{Images per Individual} \\
    \cmidrule{2-4}
    \cmidrule{7-8}
    Name & Left View & Right View & Total & Individuals & Singletons & Avg. & Max. \\
    \midrule
    Grand & 929 & 942 & 1871 & 206  & 69 & 9 & 31 \\
    Otago & 903 & 913 & 1816 & 221  & 58 & 8 & 24 \\
    \bottomrule
  \end{tabular}
\end{table}


\section{General}

\subsection{Selfscore and Score Distributions}

Sloop ranks the likelihood that two capture events contain the same individuals quantatively by a score. The similarity score between two individuals is obtained from comparing sift features of all the images within a capture capture to sift features of the images within the other individual's cohort \emph{normalized by the minimum selfscore among all the images in the comparison}. A capture is a group of images containing the same individual, not necessarily having the same view, that is captured together, whereas a cohort is a group of images that are identified as the same animal, not necessarity having the same view or captured at the same time.

\subsubsection{Selfscores}

Selfscore of a capture is the \emph{maximum} selfscore of all the images within the capture, where a selfscore of an image is calculated from comparing the sift features of each image with itself.
$$\texttt{selfscore}(C) = \max_{\forall I_i \in C} \texttt{sift\_match}(I_i, I_i)$$ 

\begin{figure}[h!]
  \centering
  \subfloat[Selfscores]{\includegraphics[width=0.58\textwidth]{dataset/grand/selfscores}}\qquad
  \subfloat[CDF of selfscores]{\includegraphics[width=0.3\textwidth]{dataset/grand/cdf_selfscores}}
  \captionsetup{justification=centering}
  \caption{Selfscore Distribution}
\end{figure}

\begin{figure}[h!]
  \centering
  \subfloat[Selfscores]{\includegraphics[width=0.58\textwidth]{dataset/otago/selfscores}}\qquad
  \subfloat[CDF of selfscores]{\includegraphics[width=0.3\textwidth]{dataset/otago/cdf_selfscores}}
  \captionsetup{justification=centering}
  \caption{Selfscore Distribution}
\end{figure}

\subsubsection{Scores}

The score between the two individuals is the maximum score among all the comparisons among the images within the cohorts.

$$\texttt{score}(C_1, C_2) = \max_{\forall I_i \in C_1 \forall I_j \in C_2} \frac{\texttt{sift\_match}(I_i, I_j)}{\min(\texttt{selfscore}(C_1), \texttt{selfscore}(C_2))}$$
where $C$ is a capture and $I$ is an image in a capture.

\begin{figure}[h]
  \centering
  \includegraphics[width=\textwidth]{dataset/grand/scores}
  \caption{Score Distribution}
  \label{fig:overview}
\end{figure}

\begin{figure}[h]
  \centering
  \includegraphics[width=\textwidth]{dataset/otago/scores}
  \caption{Score Distribution}
  \label{fig:overview}
\end{figure}
% Relevance Feedback: Relevance feedback accelerates recall
% Methodology
% * Design and Architecture
% * Population of interest and sampling subject used in the study
% * Instrument and what it measures (metrices)
% * qualifications of informants if used in the study
% * Validation
% * Data gathering procedure (experiments)
\graphicspath{{./images/chap4/}}
% Relevance Feedback
% Methodology
% * Design and Architecture
% * Population of interest and sampling subject used in the study
% * Instrument and what it measures (metrics)
% * qualifications of informants if used in the study
% * Validation
% * Data gathering procedure (experiments)
\chapter{Relevance Feedback}
\label{chap:relevance_feedback}

To study and understand the effect of relevance feedback to the retrieval
results, we implement a simulator that reproduces the relevance feedback
processes of sloop. The implementation of the simulator is described in
Section~\ref{sec:method}. We then compare the initial ranked result set
outputted from SIFT computation with the results that undergo different rounds
of relevance feedback with various configurations in
Section~\ref{sec:experiments}.

\section{Methodology}
\label{sec:method}

The relevance feedback simulator mimics the behavior of the task submission
process by the repeatedly sampling a batch of capture pairs from the pool of
all unknown capture pairs. For all the sampled pairs within a batch, the
simulator marks them as matching pairs or non-matching pairs with the
\emph{correct} answers.

When all the pairs in a specific batch are marked, it constructs an
\emph{identity graph} that represents the connections between all the known
pairs using the marked answers. The identity graph is undirected. The nodes in
the graph represent captures. If there is an edge connecting between node $A$
and $B$, capture $A$ and $B$ contain the same individual. Graphically mapping
the relationship of individuals results in fully-connected subgraphs formed by
all the captures containing the same animals.

Upon the retrieval of the new information from each batch, the simulator uses
the identity graph to infer the answers of the unknown pairs. This imitates the
in-database merging logic of the current version of Sloop~\cite{sloopdocs}. The
transitive relations are drawn from both matching pairs and non-matching pairs.
For example, given capture $A$ is known to contain the same individual as that
in capture $B$, and capture $B$ is known to contain the same individual as that
in capture $C$, the simulator can infer that capture $A$ contains the same
individual as that in capture $C$, without having known the actual answer of
$A$ and $C$. Once such transitive inference is made, it adds $C$ to the
identity graph. Then, it continues to sample a new batch iteratively until the
distribution satisfies the convergence condition specified by the user.

The correctness of such inference relies on the validity of the premises, i.e.\
if the answers to the matches are correct then the conclusion must also hold.
Problems arise when our seemingly \emph{correct} answers are, in fact,
incorrect. With the auto merging, the error propagates rapidly. An incorrect
answer yield erroneous results not only to the pair of captures in question,
but to \emph{all} the captures from the same cohorts as the captures in
question. However, incorrect answer can be  detected once inconsistency among
the answers emerges. In the actual Sloop production, all the captures within
the cohorts related to the captures in which inconsistency apears are going to
be send for expert manual review for reevaluation. On the other hand, in our
simulation, we would like to examine the consequences of the incorrect answers.
Therefore, the simulator also allows users to specify the error probability at
the initialization.

\section{Metrics} % (fold)
\label{sec:metrices}

This section describes the metrics used to evaluate the performance our
models. Along with our goal to maximize the precision and recall, we would like
to minimize the cost of the crowdsourced feedback or to operate optimally within a
budget constraint. The simulator also provides a counter that counts the number
of iterations a distribution takes to converge. However, this is out of our
interest because the time a model takes to converge can also be inferred from
the number of task submitted. Given a model, the fewer assignments posted, the
faster the model converges. 

\subsection{Precision and Recall} % (fold)
\label{sub:precision_and_recall}

As you can see, in the system where true negatives (true non-matching pairs)
largely outnumbers the true positives (true matching-pairs), comparing the
\emph{true positive rate (TPR)} or recall to the \emph{false positive rate
(FPR)} is not very meaningful. This is because false positive rate, which is
$\Pr{(\hat{y}=1|y=0)} = \frac{FP}{FP+TN}$, is going to be approximately 0 as
$TN$ is very large. Thus, instead of using ROC (receiver operating
characteristic) curve, which is a plot of TPR and FPR, we use
\emph{precision-recall curves (PR)} to analyze the performance of our models.

A precision recall curve is a plot of precision and recall as we  vary the
threshold $\theta$. Precision or positive predictive value (PPV) is defined as
following~\cite{manning2008introduction}: $$PPV = \Pr{(y=1|\hat{y}=1)} =
\frac{TP}{\hat{P}} = \frac{TP}{TP+FP}$$

We evaluate the ranked retrieval performance of different relevance feedback
sampling policies using the \emph{Mean Average Precision (MAP)}, which is
roughly the area under the \emph{precision-recall curves}.
% subsection precision_and_recall (end)

\subsection{Cost} % (fold)
\label{sub:cost}

The total amount of money we have to spend to reward the workers is
proportional to the \emph{number of tasks we published}. If, at the moment we
have just finished marking all the unknown pairs, there are still some leftover
tasks on the crowdsourcing, those tasks are not going to be cancelled (unless
the user forcefully cancels the tasks himself.) Hence, we can use the number of
tasks we published as our cost metric.
% subsection cost (end)

% section metrics (end)

\section{Experiments} % (fold)
\label{sec:experiments}

In this section, we consider a setting in which time evolves in rounds. In each
round, the we, the requester, simulates the task submission process in a
crowdsourcing platform assuming that all the workers completes all the task.
Our goal as the requester is to maximize the Mean Average Precision value of a
preliminary score distribution outputted from Sloop with the amount payments we
have to make in mind.

We report three experiments corroborating the improvement of the retrieval
results after various rounds of relevance feedback and different sampling
strategies.  Experiment~\ref{sub:batch_sizes} establishes that relevance
feedback improves the accretion of precision and recall of the preliminary
results at various batch sizes.  Experiment~\ref{sub:errors} shows the
consequence of the erroneous answers at different error probabilities.
Experiment~\ref{sub:sampling_policies} compare the performance of various
sampling policies.

All experiments are performed on the Grand and Otago dataset described in
Chapter~\ref{cha:datasets}. Both datasets are annotated by the biologist, which
we take as our true labels. Table 

\begin{table}[t]
\captionsetup{justification=centering}
  \caption{Number of capture pairs for each species}
  \label{species-num-pairs} %chktex 24
  \centering
  \begin{tabular}{lc}
    \toprule
    Species & Number of pairs \\
    \midrule
    Grand & 507528 \\
    Otago & 492690 \\
    \bottomrule
  \end{tabular}
\end{table}

\subsection{Batch Sizes} % (fold)
\label{sub:batch_sizes}

This experiment first shows that relevance feedback improves the accretion of
precision and recall of the preliminary results at different iterations and
batch sizes. To establish these relations, we randomly sample some unknown
pairs from our unknown pair pool using \emph{uniform} sampling policy, in which
drawing each unknown pair is equally probable. The reason we use uniform random
sampling policy is because it is simple and unbiased, which is useful to get a
general baseline value.

During the sampling process, we observe the Mean Average Precision (MAP) at
each iteration. A batch of samples is iteratively drawn until we know the
answers to \emph{all} the unknown pairs or MAP is equal to 1, whichever happens
first.

We then repeat the process for various batch sizes, which is defined to be the
total number of unknown pairs marked in between a pair of the following events:
initialization, identity inference, and termination. Finally, we compare the
number of pairs sampled to reach convergence for each batch size.
% subsection batch_sizes (end)

\subsection{Errors} % (fold)
\label{sub:errors}

Despite the tasks distribution and the gold standard questions, there is
probability of 0.078 of obtaining an incorrect answer, assuming that all the
preventive events are independent. However, in practice, errors occur
haphazardly rather than systematically. Thus, 0.078 is the worst case, where we
assumed the worker always make a guess with equal probability.

This experiment models such error. Given an error probability of $\epsilon$, if
a pair of captures is sampled, the simulator marks it with the annotated answer
(correct answer) with the probability of $1-\epsilon$; otherwise, the pair is
marked with the opposite label (incorrect answer). Again, we compare the values
of MAP at each iteration and the total number of pairs sampled for each error
probability.
% subsection errors (end)

\subsection{Sampling Policies} % (fold)
\label{sub:sampling_policies}

Sampling policy plays a significant role in the retrieval. Given an initial
ranking, a pair of captures whose score is within a certain range or higher
than certain threshold is more likely to be a match. Such range and threshold
is species-specific and relies on Sloop's classification performance.

We experiment with following policies:
\begin{description}
  \item [Uniform]
  Sample an unknown pair from a uniform distribution where drawing each
    \emph{pair} is equally probable.
  \item [UniformScore]
  Sample an unknown pair from a uniform score distribution where drawing a pair
    with each \emph{score} is equally probable.
  \item [TopMatches]
  Always select an unknown pair with the highest score.
  \item [Nonmatches]
  Always select an unknown pair with the lowest score.
  \item [Normal]
  Sample an unknown pair from a normal distribution with $\mu=$median and
    $\sigma=0.3$.
  \item [Percentile]
  Always select an unknown pair at the median.
  \item [AllScores]
  Divide the scores into $n$ bins, where $n=$\texttt{batch\_size} and then
  select some unknown pairs from all the bins so that the total number of
  unknown pairs sums to \texttt{batch\_size}.
\end{description}

% subsection sampling_policies (end)

% section experiments (end)

\section{Results and Discussion} % (fold)
\label{sec:results}

Figure~\ref{pr_curves} displays the Precision-Recall graph at a given number of
sampled unknown pairs. With zero error probability, relevance feedback reduces
the number of comparisons required for each specifies by a factor of 317 and
307 for grand and otago respectively. Within four iterations of feedback loop,
Mean Average Precision (MAP) reaches 1.0.

The results corroborates the fact that the relevance feedback dramatically
accretes precision and recall given the correct feedback information. Such
inclination of precision and recall is expected largely due to the
interpolation of the new information.

\begin{figure}[h!]
  \centering
  \subfloat[Grand]{\includegraphics[width=0.8\textwidth]{pr/grand}}\\
  \subfloat[Otago]{\includegraphics[width=0.8\textwidth]{pr/otago}}
  \captionsetup{justification=centering}
  \caption{Precision-Recall graph ($Error=0$, \texttt{batch\_size}$=400$)}
  \label{pr_curves} %chktex 24
\end{figure}

\subsection{Batch Sizes} % (fold)
\label{sub:batch_sizes_res}

Overall, from the graphs, MAP increases as we feed more data into the system.
Figure~\ref{fig:sizes_curves} indicates that we publish more unnecessary
tasks as we increase the batch size. The fewer captures there are in a batch,
the lower the overall cost. Taking this idea one step further, we would like to
infer the matches and merge the individuals as frequently as possible. However,
empirically, smaller batch size may upset some workers who would like to
continuously work on the tasks. Therefore, we need to find the smallest batch
size that is still large enough to engage the workers.

\begin{figure}[h!]
  \centering
  \subfloat[Grand]{\includegraphics[width=0.8\textwidth]{sizes/graoc}}\\
  \subfloat[Otago]{\includegraphics[width=0.8\textwidth]{sizes/otaoc}}
  \captionsetup{justification=centering}
  \caption{Mean Average Precision (MAP) for different batch sizes ($Error=0$)}
\end{figure}

\begin{figure}[h!]
  \centering
  \subfloat[Grand]{\includegraphics[width=0.8\textwidth]{sizes/grtotal}}\\
  \subfloat[Otago]{\includegraphics[width=0.8\textwidth]{sizes/ottotal}}
  \captionsetup{justification=centering}
  \caption{The total number of unknown pairs marked to achieve $MAP=1$
  ($Error=0$, \texttt{batch\_size}$=100$). Notice that this equals the total 
  number of tasks published to achieve $MAP=1$. }
  \label{fig:sizes_curves} %chktex 24
\end{figure}

% subsection batch_size (end)

\subsection{Errors} % (fold)
\label{sub:errors}

Despite the present of errors, relevance feedback still yields higher MAP
overall with an error threshold of 0.05. With a nonzero error probability, MAP
curves downward before it curves up and reach a saturation point. This is
because initially when it does not have much data, the system is very sensitive
to errors, especially the false negatives, which trigger the irreversible merge
operation. Thus, the precision decline rapidly.  As it gains more correct data,
it is able to recover from the downward phase. However, the historical merges
resulted from the past faulty data are irrevocable, so the precision saturates
eventually.

\begin{figure}[ht]
  \centering
  \subfloat[Grand]{\includegraphics[width=\textwidth]{errors/graoc}}
  \caption{Mean Average Precision for different probabilities of error}
  \label{fig:grand_aoc} %chktex 24
\end{figure}
% subsection errors (end)

\subsection{Sampling Policies} % (fold)
\label{sub:sampling_policies_res}

Uniform sampling (Uniform), normal sampling (Normal), and sampling from all the
scores (AllScores) seem to outperform other sampling policies in term of the
total cost. The samplers that sample single score values at a time, such as
sampling from median (Percentile), performs poorly compare to others. However,
this depends largely on the species of the animal that we are interested in. As
you can see the performance of the each sampling policy differs slightly
between grand and otago.

\begin{figure}[h!]
  \centering
  \subfloat[Grand]{\includegraphics[width=0.8\textwidth]{policies/grand}}\\
  \subfloat[Otago]{\includegraphics[width=0.8\textwidth]{policies/otago}}
  \captionsetup{justification=centering}
  \caption{Mean Average Precision (MAP) for various sampling policies ($Error=0$,
    \texttt{batch\_size}$=100$)}
\end{figure}
% subsection sampling_policies (end)

% section results (end)

% Sloop-mturk: Crowdsourced relevance feedback on mturk
% Methodology
% * Design and Architecture
% * Population of interest and sampling subject used in the study
% * Instrument and what it measures (metrices)
% * qualifications of informants if used in the study
% * Validation
% * Data gathering procedure (experiments)
\graphicspath{{./images/chap5/}}
% Sloop-mturk: Crowdsourced relevance feedback on mturk
% Methodology
% * Design and Architecture
% * Population of interest and sampling subject used in the study
% * Instrument and what it measures (metrics)
% * qualifications of informants if used in the study
% * Validation
% * Data gathering procedure (experiments)
\chapter{Crowdsourced Relevance Feedback}
\label{chap:sloop_mturk}
% Sloop-mturk: Crowdsourced relevance feedback on mturk
\section{Sloop MTurk}

As described in Chapter~\ref{chap:relevance_feedback}, multiple rounds of
relevance feedback accretes precision and recall. This
chapter presents the architecture of \emph{Sloop MTurk}, the crowdsourced
relevance feedback engine for Sloop.

\begin{figure}[htb]
  \centering
  \includegraphics[width=0.8\textwidth]{sloop/turk_system}
  \caption{Sloop Architecture with Sloop MTurk integration}
  \label{fig:turk_overview} %chktex 24
\end{figure}

Sloop MTurk communicates with the crowdsourcing platform in order to obtain
human feedback on the original rankings. Our focus is on

\subsection{Architecture}

There are four major functions used within the relevance feedback
workflow: Retrieve, Publish, Fetch, and Update.

\subsubsection{Retrieve}

In the first stage of the relevance feedback workflow, Sloop MTurk retrieves a
number of known and unknown image pairs from the database in the ratio
described in Section~\ref{subsub:validation}. The metadata about the retrieved
pairs, such as the answers to the known pairs, individual identification
numbers, and source address, is stored in a lightweight SQLite database.

Sloop MTurk never publishes a duplicated \emph{unknown} pair onto MTurk. It
actively checks with the local database before any retrieval whether an unknown pair
has ever been published before. Users can set the sampling policy of Sloop MTurk in
the configuration file. The default is to use the normal sampling from the
median whose performance has been shown in
Chapter~\ref{chap:relevance_feedback}.

Sloop MTurk only fetches from the Sloop database the images necessary for the
tasks to be published.  Upon retrieval, Sloop MTurk caches the image data
locally on the filesystem behind NGINX\@. Caching image data reduces data transfer
overhead because fetching large responses (like images) over the network is
both slow and expensive.

\subsubsection{Publish}

Sloop MTurk publishes the tasks to MTurk immediately after all the
information is fetched. The tasks published are available on Amazon Mechanical
Turk under the title: `Image Matching --- Animals' posted by user `sloop'.
% TODO Tell user that they need AWS access ID and secret key to
% programmatically communicate with MTurk

\subsubsection{Fetch}

Users can fetch results once the published tasks have been completed. This can
be run as a cron job.  For each completed task with correct answers to all
known pairs, Sloop MTurk accepts and retrieves the results, allowing the worker
to be paid.  For any tasks with incorrect answers to known pairs, Sloop MTurk
rejects the results, with no payment given.

Sloop MTurk logs the answers to the unknown pairs from the accepted assignments
in the local SQLite database. Results are never deleted automatically.

\subsubsection{Update}

The update command pushes the answers logged in the local SQLite database back
to the original Sloop database so that the data is ready for another round of
relevance feedback. The remote database server then categorizes the captures
with the data pushed and the existing data, and then performs its cohort
merging logic~\cite{sloopdocs}.

The update command is separated from fetch for ease of debugging. In
practice, we would like to update the upstream database as frequently as
possible as we have seen in the Experiments section,
Chapter~\ref{chap:relevance_feedback}.

\subsection{Platform Selection}

Workers and requesters interact via a crowdsourcing platform. All our tasks
are published on Amazon Mechanical Turk (MTurk), which is one of the most
popular crowdsourcing platforms.

\subsection{Human Intelligence Task (HIT)}

Tasks published on MTurk are called Human Intelligence Tasks (HITs). The terms
`HIT' and `tasks' will be used interchangeably throughout this report.

Requesters publish the tasks that they need completed to Amazon Mechanical
Turk's marketplace with a set payment to reward workers who finish the tasks.

\subsubsection{Task Design}

A wide variety of tasks can be crowdsourced. There are two possible designs we
have considered:
\begin{enumerate}
	\item Rankings and weights \\
    Each worker may be asked to provide a full ranking or relative weights
    for a given pool of candidates. Although knowing the correct rankings is
    valuable to ranked retrieval results,
    ultimately, we would like to determine a sharp cutoff between matching
    images and non-matching images so that each individual can be exclusively
    identified. Ranks alone do not provide such a cutoff. Even if we knew all the
    correct rankings, we would not be able to decide whether two animals are
    the same individual. Thus, rankings and weights is not a viable
    solution.
	\item Multiple-choice questions \\
    A question can include multiple options. For instance, asking which of the images
    in the choices contain an animal that matches the given individual
    animal. However, this approach only tells us which of the pictures among
    all the choices best matches a given image, which is an inefficient
    approach to ranking images. Alternatively, we can ask workers to provide
    all the matches among all the options. In this case, there can be none or
    more than one answer. However, using this method, we only gain the
    information of one individual from $n$ comparisons, where $n$ is the number
    of choices for each question.

    We can model our task as a binary question of whether a pair of images from
    the same view contain the same individual animal (Is a match?).
    Eventually, all the image pairs will be labelled as either a match or a
    non-match, and we will be able to construct a mapping that allows us to
    exclusively identify all the individual animals in the pictures.
\end{enumerate}

\subsubsection{Validation}
\label{subsub:validation}

For each task, since the quality of the submitted work is not directly
observable, we add three other \emph{gold standard} questions, for which the
correct answers are known \emph{a priori}. Among the three questions, one is a known
pair of images containing the same individual, another is a known pair of
images containing different individuals, and the other is a random known pair.
These \emph{gold standard} questions work as qualifying tests for eligible
workers. Additionally, they also accelerate the workers' learning process and
skill level.

To further ensure the correctness of the answers submitted by the workers, we
publish the same assignment to three workers, and determine the correct answer
by consensus. We only allow workers with a task acceptance rate higher than
0.8, to prevent spam. If the majority agrees on some answer, it may be safe to
assume that it is correct. Obviously, the more number of workers we assign to a
given task, the less chance there is of getting an incorrect answer. However,
the budget and the time it takes to complete all the assignments increases
linearly as we increase the number the workers, while the
correctness only increases logarithmically. Therefore, we decided to publish
exactly three assignments, which is the minimum number required to reach a consensus
for each task.

\subsection{Cost Model}

Sloop MTurk uses the evaluation metrics mentioned in
Chapter~\ref{chap:relevance_feedback} to measure the performance of the system.

\subsubsection{Correctness}

To analyze the correctness, we still use the Mean Average Precision.  MAP has
been shown to have especially good discrimination and stability for evaluating
information retrieval
systems~\cite{manning2008introduction}.

\subsubsection{Expense}

Upon completion of a task, each worker receives the posted price for that
task. The more tasks published, the higher the total payment we have to make.

% Adaptive sampling method
% Methodology
% * Design and Architecture
% * Population of interest and sampling subject used in the study
% * Instrument and what it measures (metrices)
% * qualifications of informants if used in the study
% * Validation
% * Data gathering procedure (experiments)
\graphicspath{{./images/chap6/}}
% Adaptive sampling method
% Methodology
% * Design and Architecture
% * Population of interest and sampling subject used in the study
% * Instrument and what it measures (metrics)
% * qualifications of informants if used in the study
% * Validation
% * Data gathering procedure (experiments)
\chapter{Relevance Feedback with Adaptive Sampling}

One of the most important questions we would like to answer is: given an
initial distribution of scores, where should we be sampling from? In this
chapter, we take advantage from the fact that we gradually gain more
information about the score distribution over time to dynamically determine
where we should be sampling from.

First, we perform different statistical analyses on the data. Then we propose a
new sampling scheme based on our analyses. Finally, we set up an experiment to
compare the performance of our proposed scheme to the other standard sampling
policies we have implemented.

\section{Statistical Analysis}

\subsection{Probability of Score given matches/non-matches}

We can estimate the probability of score given a match by
$$\Pr{(score=s \mid match)} = \frac{\texttt{\# matches\_whose\_score=s}}
    {\texttt{\# matches}}$$
In the same way,
$$\Pr{(score=s \mid nonmatch)} = \frac{\texttt{\# nonmatches\_whose\_score=s}}
    {\texttt{\# nonmatches}}$$

\begin{figure}[h!]
  \centering
  \subfloat[][Probability of Score given matches/non-matches]
  {\includegraphics[width=0.95\textwidth]{dataset/grand/psm}}
  \label{fig:grand_psm}\\ %chktex 24
  \subfloat[][CDF of Score given matches/non-matches]
  {\includegraphics[width=0.95\textwidth]{dataset/grand/csm}}
  \label{fig:grand_csm}\\ %chktex 24
  \subfloat[][Difference between the CDF of Score given matches/non-matches]
  %$F\[non-match\]$ and $F\[match\]$
  {\includegraphics[width=0.95\textwidth]{dataset/grand/dsm}}
  \caption{Functions of Score given matches/non-matches}
  \label{fig:grand_dsm} %chktex 24
\end{figure}

From the plot above, we can deduct following conclusions:
\begin{itemize}
\item Most of the cohorts has score between 0 and 0.07 overall so most matches
and non-matches have score less than 0.06.
\item We expected $\Pr{(score \mid match)}$ to increase as the function of score
, but it turns out that it actually decreases. This results from the fact that
there are \emph{very few} matches with scores higher than 0.33, which is the 99th
percentile of the matches' score.
\item The number of matches with $score \in [0.08, 0.35)$ is greater than that of
non-matches with the score in the same range. That is
$$\Pr{(score \mid match)} > \Pr{(score \mid nonmatch)} \mbox{ where } score\in [0.08, 0.35)$$
\end{itemize}

\begin{figure}[h!]
  \centering
  \subfloat[][Probability of Score given matches/non-matches]
  {\includegraphics[width=0.95\textwidth]{dataset/otago/psm}}
  \label{fig:otago_psm}\\ %chktex 24
  \subfloat[][CDF of Score given matches/non-matches]
  {\includegraphics[width=0.95\textwidth]{dataset/otago/csm}}
  \label{fig:otago_csm}\\ %chktex 24
  \subfloat[][Difference between the CDF of Score given matches/non-matches]
    %$F\[non-match\]$ and $F\[match\]$
  {\includegraphics[width=0.95\textwidth]{dataset/otago/dsm}}
  \caption{Functions of Score given matches/non-matches}
  \label{fig:otago_dsm} %chktex 24
\end{figure}


\subsection{Probability of a matching score at a given cohort size}

\begin{figure}[ht]
  \centering
  \subfloat[][Probability of a matching score at a given cohort size]
  {\includegraphics[width=0.95\textwidth]{dataset/grand/pscohort}}
  \label{fig:grand_pscohort}\\ %chktex 24
  \subfloat[][Probability of a score \emph{not} in a given cohort size]
  {\includegraphics[width=0.95\textwidth]{dataset/grand/psnoncohort}}
  \caption{Probability of a score given a range of cohort size}
  \label{fig:grand_psnoncohort} %chktex 24
\end{figure}

Initially, we expected that as the cohort size increases, $\Pr{(score \mid
\#cohort)}$ should peak closer to 1 since $\texttt{selfscore}$ should decrease
and $\max{(\texttt{sift\_match}(I_i, I_j))}$ should increase as the number of
captures in a cohort increases. However, the distribution of $\Pr{(score \mid
\#cohort)}$ for all numbers of cohorts seem to be similar to one another.

\begin{figure}[ht]
  \centering
  \subfloat[][Probability of a matching score at a given cohort size]
  {\includegraphics[width=0.95\textwidth]{dataset/otago/pscohort}}
  \label{fig:otago_pscohort}\\ %chktex 24
  \subfloat[][Probability of a score \emph{not} in a given cohort size]
  {\includegraphics[width=0.95\textwidth]{dataset/otago/psnoncohort}}
  \caption{Probability of a score given a range of cohort size}
  \label{fig:otago_psnoncohort} %chktex 24
\end{figure}

\subsection{Earth Mover's Distance}

We use Earth Mover's Distance (EMD)~\cite{emd00} as our metric to calculate the
minimal cost required to transform $\Pr{(score \mid nonmatch)}$ to $\Pr{(score \mid match)}$.

\begin{figure}[ht]
  \centering
  \includegraphics[width=0.95\textwidth]{dataset/grand/emd}
  \caption{Earth Mover's Distance between the probability distributions}
  \label{fig:grand_emd} %chktex 24
\end{figure}

Initially, we also expected that the relationship between EMD and cohort size
should be Gaussian. However, there seem to be no relationship between
$\Pr{(score \mid nonmatch)}$ and $\Pr{(score \mid match)}$.

\begin{figure}[ht]
  \centering
  \includegraphics[width=0.95\textwidth]{dataset/otago/emd}
  \caption{Earth Mover's Distance between the probability distributions}
  \label{fig:otago_emd} %chktex 24
\end{figure}

\subsection{Entropy of $\Pr{(score \mid match)}$ at different cohort sizes}

Entropy is the average amount of information we get from each distribution.
$H(X|Y)$ = amount of randomness in the random variable X given the event Y.

\begin{figure}[ht]
  \centering
  \subfloat[][Entropy of $\Pr{(score \mid match)}$ at different cohort sizes]
  {\includegraphics[width=0.95\textwidth]{dataset/grand/ent_psm}}
  \label{fig:grand_ent_psm}\\ %chktex 24
  \subfloat[][Entropy of $\Pr{(score \mid nonmatch)}$ at different cohort sizes]
  {\includegraphics[width=0.95\textwidth]{dataset/grand/ent_psnm}}
  \caption{Relationship between $\Pr{(score \mid match)}$ and $\Pr{(score \mid nonmatch)}$}
  \label{fig:grand_ent_psnm} %chktex 24
\end{figure}

\begin{figure}[ht]
  \centering
  \subfloat[][Entropy of $\Pr{(score \mid match)}$ at different cohort sizes]
  {\includegraphics[width=0.95\textwidth]{dataset/otago/ent_psm}}
  \label{fig:otago_ent_psm}\\ %chktex 24
  \subfloat[][Entropy of $\Pr{(score \mid nonmatch)}$ at different cohort sizes]
  {\includegraphics[width=0.95\textwidth]{dataset/otago/ent_psnm}}
  \label{fig:otago_ent_psnm}\\ %chktex 24
  \caption{Relationship between $\Pr{(score \mid match)}$ and $\Pr{(score \mid nonmatch)}$}
  \label{fig:emd_grand} %chktex 24
\end{figure}

\subsection{Probability of a Matches/Non-matches}

$$\Pr{(match \mid score)} = \frac{\Pr{(score \mid match)}\Pr{(match)}}
    {\Pr{(score)}}$$

$$\Pr{(nonmatch \mid score)} = \frac{\Pr{(score \mid nonmatch)}\Pr{(nonmatch)}}
    {\Pr{(score)}}$$

\begin{figure}[ht]
  \centering
  \includegraphics[width=0.95\textwidth]{dataset/grand/pms}
  \caption{Probability of matches/non-matches given score}
  \label{fig:grand_pms} %chktex 24
\end{figure}

\begin{figure}[ht]
  \centering
  \includegraphics[width=0.95\textwidth]{dataset/otago/pms}
  \caption{Probability of matches/non-matches given score}
  \label{fig:otago_pms} %chktex 24
\end{figure}

\section{Adaptive Sampling Policies} % (fold)
\label{sec:sampling_policies}

Starting with a completely unknown pairs, we would like to first get the idea
of how the overall distribution looks like. As we have seen in
Chapter~\ref{chap:relevance_feedback}, the unbiased sampling policies, such as
Uniform, and AllScores, that broadly sample the data from the distribution seem
to form a set of good candidates.

Upon the retrieval of the general distribution, we would like to adaptively
utilize the known data to determine the optimal sampling site. We experiment
with sampling from the peak of the difference between the CDF of the score
given matches and non-matches using various sampling policies.

After we get the general picture of the distribution, we repeatedly reconstruct
the plot of the difference between the CDF of the score given matches and
non-matches, then uses normal sampling policy to overlay a Gaussian
distribution, with $\mu=$\texttt{peak\_of\_the\_CDF\_difference} and
$\sigma=0.3$, on the data. We continue each cycle by the sampling probability
distribution reconstruction, and sampling from the generated distribution.
Additionally, we also use Specific sampling policy to sample specifically at the
\texttt{peak\_of\_the\_CDF\_difference} to compare the performance with the
Normal sampling policy.

\section{Results}

\begin{figure}[h!]
  \centering
  \includegraphics[width=0.8\textwidth]{otago}
  \caption{Mean Average Precision for the new sampling policies}
  \label{fig:otago_adaptive_aoc} %chktex 24
\end{figure}

The adaptive sampling policy seem to perform averagely well. The performance
does not differ from that of the other static sampling policies we have seen in
Chapter~\ref{chap:relevance_feedback}.

% section proposed_sampling_policy (end)

% Classification with CNN
% Methodology
% * Design and Architecture
% * Population of interest and sampling subject used in the study
% * Instrument and what it measures (metrices)
% * qualifications of informants if used in the study
% * Validation
% * Data gathering procedure (experiments)
\graphicspath{{./images/chap7/}}
% Classification with CNN
% Methodology
% * Design and Architecture
% * Population of interest and sampling subject used in the study
% * Instrument and what it measures (metrics)
% * qualifications of informants if used in the study
% * Validation
% * Data gathering procedure (experiments)
\chapter{Convolutional Neural Network} % (fold)
\label{cha:convolutional_neural_network}

We first describe our new architecture in System Overview
(Section~\ref{sec:system_overview}). Then in Experiments
(Section~\ref{sec:experiments}), we compare the performance of the current
identification method within the existing pattern retrieval engine with that of
the new architecture.

\section{System Overview} % (fold)
\label{sec:system_overview}

In this section, we begin by presenting the architecture that outputs a
yes-or-no answer to the question whether, given two images A and B, the animal
in image A is the same individual as the animal in image B.

The proposed architecture comprises two major modules shown in
Figure~\ref{fig:cnn_overview}.

\begin{figure}[htb]
  \centering
  \includegraphics[width=\textwidth]{system/overview}
  \caption{System Overview}
  \label{fig:cnn_overview} %chktex 24
\end{figure}

\begin{description}
    \item[Feature Extraction] We use a pre-trained CNN with the last
        fully-connected layer (the output layer) removed to extract fixed-size feature
  vectors from the input images before feeding them into the \emph{Match
  Recognition} module.
  \item[Match Recognition]
\end{description}

\subsection{Feature Extraction} 

For a pattern retrieving engine such as Sloop, the system continually
accumulates more data over time. The availability of abundant data enables
learning based methods to outperform engineered features because they can
discover and optimize features for the specific task at hand.

Instead of hand-picking the features to extract similarity matrices from the
data, a convolutional neural network (CNN) will be used as a feature extractor,
similar to the solution proposed in~\cite{chopra05} for face verification.

We use a pre-trained CNN, AlexNet by Krizhevsky et al.~\cite{kriz12} trained on
ImageNet~\cite{imagenet}, which contains 1.2 million images with 1000
categories. To convert the CNN into a feature extractor, we strip out the last
fully-connected layer, which in this case, outputs 1000 class scores for the
ImageNet classification task. According to the AlexNet architecture, this would
output a sparse 4096 dimensional vector for every image. This technique is
called transfer learning~\cite{transfer}.

We map all the available images as well as the input image to points in a low
dimensional space (4096-D compared to $227 \times 227$-D) using the
aforementioned CNN architecture. 

\begin{figure}[htbp]
  \centering
  \begin{subfigure}[t]{0.45\textwidth}
      \centering
      \includegraphics[width=7cm]{preprocess/paramconv1}
      \caption{The first layer filters, \texttt{conv1}}
  \end{subfigure}
  ~\\
  \begin{subfigure}[t]{0.45\textwidth}
      \centering
      \includegraphics[width=7cm]{preprocess/blob20conv1}
      \caption{The first layer output, \texttt{conv1}}
  \end{subfigure}
  ~
  \begin{subfigure}[t]{0.45\textwidth}
      \centering
      \includegraphics[width=7cm]{preprocess/blob20pool5}
      \caption{The fifth layer after pooling, \texttt{pool5}}
  \end{subfigure}
  ~ \\
  \begin{subfigure}[t]{0.45\textwidth}
      \centering
      \includegraphics[width=7cm]{preprocess/lastconv1}
      \caption{The first layer output, \texttt{conv1}}
  \end{subfigure}
  ~
  \begin{subfigure}[t]{0.45\textwidth}
      \centering
      \includegraphics[width=7cm]{preprocess/lastpool5}
      \caption{The fifth layer after pooling, \texttt{pool5}}
  \end{subfigure}
  \captionsetup{justification=centering}
  \caption{Visualizing Convolutional Network layers}
\end{figure}


\textbf{Pre-trained Convolutional Neural Network}

The main advantage of using a pre-trained CNN for our application is that it is
robust to geometric distortion. This is a desirable property since the position
of the individuals in our dataset is not aligned. This enables us to
accurately recognize the animal regardless of its position in the image. 

The translation invariant property emerges as a result of the contiguity in our
pooling regions and the fact that we only pool features generated from the same
hidden units~\cite{ufldl}. Even though this property is desirable to reduce the
translation noise, this prevents learning some features that are described by
their relative positions. For example, individual $A$ with a star-shaped spot
right above its eye may be confused with another individual $B$ who also has
star-shaped spot underneath its eye. 

The solution to this problem is to experiment with different patterns of pooling
regions and relax the parameter sharing scheme. Since we are using an existing
pre-trained architecture, it is hard to tailor it to fit our data. This improvement
can be included in future work (see Chapter~\ref{cha:future_work}).

\textbf{Caffe}

The CNN is implemented in Python using Caffe~\cite{caffe}, a deep learning
framework developed by the Berkeley Vision and Learning Center (BVLC).

\subsection{Match Recognition}

Once we have the 4096-D vectors for all images generated with our CNN feature
extractor, we transfer them into a second target match recognizer and train it
on a target dataset and task. The match recognizer is a linear classifier that,
given a pair of image vectors, decides whether the two images contain the same
individual.

\textbf{Input Processing}

Upon receiving the input from the feature extractor, we process the input using
one of two methods before passing it into the recognizer, which outputs a
binary label determining whether the pair (A, B) is a match.

\begin{enumerate}
\item Concatenation

Given two input image vectors $\vec{A}$ and $\vec{B}$ (flattened), we
concatenate them horizontally so that the output $\vec{E}$ becomes $\vec{E} =
[ \vec{A}\, \vec{B} ]$. To make the order deterministic, we always start with
the vector with the smaller sum. For equal sums, we compare each pair of
corresponding entries in both vectors until we find an entry with a smaller
element.

We can visualize this as feeding a tuple of image vectors $(\vec{A}, \vec{B})$
to a linear classifier. This is equivalent to giving the classifier all the
information we have and expecting it to figure out the optimal parameters.
However, in this case there is no separator indicating that two image vectors
are actually separated.

\item Computing the absolute difference

We would like to find a similarity metric that represents how much two images
differ from each other. In order to achieve that we came up with following
metric:
$$\vec{E} = |\vec{A} - \vec{B}|$$
Each entry $e_i \in \vec{E}$ is
small if $\vec{A}$ and $\vec{B}$ belong to the same category, and large if
different.  Not only does this encapsulate the desired numerical behavior, but it
also eliminates the absence of the separator problem.
  
\end{enumerate}

\textbf{Recognizer}

Images can be uploaded to Sloop one by one or in a batch. One major difference
is that batch processing keeps the system weights constant while computing
the error associated with each sample in the input, whereas the on-line version
constantly updates its weights. For a real-time system like Sloop, we would
prefer the on-line version of the classification algorithm given the same
classification performance.

We have implemented the following algorithms as our match recognizer:
\begin{itemize}
  \item Linear support vector machine (SVM) with L2 regularization (squared
    Euclidean norm)
  \item SVM with radial basis function kernel (SVM-RBF), L2 regularization
  \item Perceptron (P)
  \item Passive-Aggressive with hinge loss (PA-I)
  \item Passive-Aggressive with squared hinge loss (PA-II)
\end{itemize}

We compare the performance of each classifier in Experiments
(Section~\ref{sec:experiments}). The classifier with the best performance will
be selected for our final design.
% section system_overview (end)

\section{Experiments} % (fold)
\label{sec:experiments}

In this section, we first introduce the datasets used in the experiments, then
present a detailed evaluation of each variant of our architecture and the
comparison between the presented solutions.

\subsection{Dataset}

The training and testing data is available in the existing Sloop system. We
base our experiments on both species of skinks: \emph{Grand} and \emph{Otago}
mentioned in Chapter~\ref{cha:datasets}.

For our experiments, all images are reduced to the size of
$227 \times 227$ pixels so that it can be directly fed into the CNN
architecture to speed up feature extraction. Regardless of our preprocessing,
the CNN will resize the images into $227 \times 227$ to fit its input
dimension. However, resizing is not necessary because practically our system
should maintain its identification capability without regard to the variations
in sizes, lighting, or background.

\textbf{Partitioning}

We divide the data from both databases into four datasets by animal species:
Gr$^{L}$-I, Gr$^{L}$-II, Ot$^{L}$-I, Ot$^{L}$-II where Gr is for Grand, Ot is
for Otago, and $L$ is for the left view. Notice that the right view is not be used for the
experiments because the results should be similar to that of left view by
symmetry. In addition, due to our memory limit during the processing, each
dataset contains only 300 images of individuals whose image per individuals are
greater than 11.

For the purpose of generating a test set that resembles the empirical input and
another test set with images that are not seen during training, we build
our datasets using two different techniques.
\begin{enumerate}
  \item Gr$^{L}$-I, Ot$^{L}$-I represents the empirical input where the data in
  the test set can also be seen in the training set.
  \item Gr$^{L}$-II, Ot$^{L}$-II represents the test set with images that are
  not seen during training. The set of individuals is split into three
  disjoint sets for training, validating, and testing. Each image in a set can
  only be paired with the images within the same set.
\end{enumerate}

For each dataset, we partition the set of \emph{generated pairs} into three
sets: the train set (60\%), the validation set (20\%), and the test set (20\%).
We prevent the overfitting problem by monitoring the model's performance on a
validation set, acting as a representation of future test examples. If the
model's performance ceases to improve sufficiently on the validation set then
we test the model on the actual test set.


\begin{table}[t]
\captionsetup{justification=centering}
  \caption{Details of the train, validation, and test set for the four datasets}
\centering
  \begin{tabular}{llrcc}
    \toprule
    \multicolumn{2}{l}{Name (NTotal)}      & NPairs & \% Match & \% Non-match \\
    \midrule
    \multirow{3}{*}{Gr$^{L}$-I (10878)}  & Train & 6526 & 0.07 & 0.93 \\
                                 & Val   & 2175 & 0.07 & 0.93 \\
                          & Test  & 2177 & 0.07 & 0.93 \\
\midrule
\multirow{3}{*}{Ot$^{L}$-I (10878)}  & Train & 6526 & 0.06 & 0.94 \\
                                 & Val   & 2175 & 0.06 & 0.94 \\
                                 & Test  & 2177 & 0.06 & 0.94 \\
\midrule
  \multirow{3}{*}{Gr$^{L}$-II} & Train & 2926 & 0.13 & 0.87\\
                                 & Val   & 351 & 0.38 & 0.62\\
                                 & Test  & 1035 & 0.25 & 0.75\\
    \midrule
    \multirow{3}{*}{Ot$^{L}$-II} & Train & 3240 & 0.10 & 0.90\\
                                 & Val   & 561 & 0.31 & 0.69\\
                                 & Test  & 561& 0.26 & 0.73\\
    \bottomrule
  \end{tabular}
  \label{tab:results_tvt}
\end{table}
% section experiments (end)

% chapter convolutional_neural_network (end)

% Analysis and Interpretation of data
% * meaning
% * in the studies involving
% * interconnection among data
% * Check for indicaotr whether hypothesis is supported by he finding
% * Link present finding to previous liturature
% Analysis and Interpretation of data
% * meaning
% * in the studies involving
% * interconnection among data
% * Check for indicaotr whether hypothesis is supported by he finding
% * Link present finding to previous liturature
\chapter{Analysis and Interpretation}

\section{Convolutional Neural Network}

\subsection{Recognition}

To evaluate the initial results, we calculate precision, recall, and F1 values of the predicted results compare to the gold standard annotated by the biologists.

\textbf{Input Processing}

According to the proposed architecture in section \ref{system}, we have experimented with both method of input processing, namely, concatenation and computing the difference. The results are shown in table \ref{concatenation-table} and \ref{result-table}.


\begin{table}[t]
\captionsetup{justification=centering}
  \caption{Precision (P), recall (R), and F-score (F) of the classification results for the input processing with feature vectors concatenation}

  \label{concatenation-table}
\centering
\begin{tabular}{lllllllllllllllllll}
    \toprule
    \multicolumn{2}{c}{\multirow{2}{*}{Dataset} } & \multicolumn{3}{c}{PCT} & \multicolumn{3}{c}{PA-I} & \multicolumn{3}{c}{PA-II}\\
  \cmidrule{3-10}
                                              & & P & R & F  & P & R & F  & P & R & F \\
    \midrule
    \multirow{3}{*}{Gr$^{L}$-I}  & Train & 0.01 & 0.03 & 0.01 & 0.04 & 0.01 & 0.01 & 0.00 & 0.00 & 0.00 \\
    \cmidrule{2-11}
                                 & Val   & 0.40 & 0.13 & 0.08 & 0.23 & 0.06 & 0.08 & 0.46 & 0.08 & 0.06 \\
    \cmidrule{2-11}
                                 & Test  & 0.01 & 0.05 & 0.02 & 0.03 & 0.01 & 0.01 &  0.00 & 0.02 & 0.01 \\
    \bottomrule
  \end{tabular}
\end{table}

All the classifiers perform poorly when feature vectors concatenation is used as the input processing method. This happens because of the \emph{curse of dimensionality}, where there are too many unnecessary input features. Doubling the dimension of the input image vector overcomplicates the classifier parameters causing the downfall in the prediction accuracy. On the other hand, the classifiers tend to perform pretty well when using absolute feature difference as the input.

\afterpage{%
    \clearpage% Flush earlier floats (otherwise order might not be correct)
    \thispagestyle{empty}% empty page style (?)
    \begin{landscape}% Landscape page
    \begin{table}   \label{result-table}
      \captionsetup{justification=centering}
        \caption{Precision (P), recall (R), and F-score (F) of the linear classifier with absolute feature difference on train, validation, and test set of the four datasets}

      \centering % Center table
      \hskip-2.0cm\begin{tabular}{lllll|lll|lll|lll|lllll}
          \toprule

          \multicolumn{2}{c}{\multirow{2}{*}{Dataset} } & \multicolumn{3}{c}{SVM} & \multicolumn{3}{c}{SVM-RBF} & \multicolumn{3}{c}{PCT} & \multicolumn{3}{c}{PA-I} & \multicolumn{3}{c}{PA-II}\\
        \cmidrule{3-17}
                                                    & & P & R & F  & P & R & F  & P & R & F  & P & R & F  & P & R & F \\
          \midrule
          \multirow{3}{*}{Gr$^{L}$-I}  & Train & 0.91 & 0.21 & 0.34 & 0.97 & 0.18 & 0.30 & 0.40 & 0.26 & 0.32 & 0.57 & 0.31 & 0.29 & 0.57 & 0.29 & 0.30       \\
          %\cmidrule{2-17}
                                       & Val   & 0.95 & 0.23 & 0.37 & 0.99 & 0.19 & 0.32 & 0.58 & 0.37 & 0.45 & 0.65 & 0.48 & 0.38  & 0.67 & 0.44 & 0.39      \\
          %\cmidrule{2-17}
                                       & Test  & 0.82 & 0.20 & 0.33 & 1.0 & 0.21 & 0.35 &  0.42 & 0.24 & 0.30 & 0.54 & 0.33 & 0.28 & 0.52 & 0.29 & 0.28     \\
          \hline
          \multirow{3}{*}{Ot$^{L}$-I}  & Train & 0.64 & 0.24 & 0.32 & 0.88 & 0.21 & 0.34  & 0.45 & 0.32 & 0.29 & 0.37 & 0.16 & 0.23 & 0.36 & 0.27 & 0.30     \\
      %\cmidrule{2-17}
                                       & Val   & 0.81 & 0.42 & 0.51 & 0.90 & 0.24 & 0.37 & 0.55 & 0.54 & 0.42 & 0.89 & 0.30 & 0.39 & 0.65 & 0.55 & 0.59     \\
      %\cmidrule{2-17}
                                       & Test  & 0.66 & 0.26 & 0.33 & 0.90 & 0.20 & 0.32 & 0.46 & 0.39 & 0.33 & 0.36 & 0.17 & 0.23 & 0.42 & 0.32 & 0.36     \\
          \hline
          \multirow{3}{*}{Gr$^{L}$-II} & Train & 0.92 & 0.35 & 0.50 & 0.99 & 0.21 & 0.34 & 0.99 & 0.27 & 0.43 & 0.89 & 0.48 & 0.63 & 0.83 & 0.57 & 0.67     \\
      %\cmidrule{2-17}
                                       & Val   & 0.79 & 0.20 & 0.32 & 1.0 & 0.20 & 0.33 & 0.93 & 0.19 & 0.32 & 0.76 & 0.21 & 0.33 & 0.66 & 0.24 & 0.36     \\
      %\cmidrule{2-17}
                                       & Test  & 0.85 & 0.22 & 0.35 & 1.0 & 0.18 & 0.30 & 0.92 & 0.19 & 0.31 & 0.86 & 0.20 & 0.32 & 0.69 & 0.22 & 0.34     \\
          \hline
          \multirow{3}{*}{Ot$^{L}$-II} & Train & 0.57 & 0.29 & 0.38 & 1.0 & 0.22 & 0.38 & 0.53 & 0.70 & 0.60 & 0.59 & 0.78 & 0.67 & 1.0 & 0.25 & 0.40     \\
      %\cmidrule{2-17}
                                       & Val  & 0.57 & 0.68 & 0.62 & 1.0 & 0.24 & 0.39 & 0.53 & 0.37 & 0.43 & 0.56 & 0.30 & 0.39 & 0.97 & 0.19 & 0.32   \\
      %\cmidrule{2-17}
                                       & Test  & 0.57 & 0.28 & 0.38  & 1.0 & 0.19 & 0.32 & 0.43 & 0.36 & 0.39 & 0.50 & 0.32 & 0.39 & 1.0 & 0.22 & 0.37     \\
          \bottomrule
        \end{tabular}
      \end{table}
    \end{landscape}
    \clearpage% Flush page
}



Considering the classification results in \ref{result-table}, all the classifiers seem to perform equally well on the test data considering the F-beta score. SVM with radial bas kernel function yields the most stable performance with very high precision for both species. The offline algorithms slightly outrun all the online ones on average.

In the system such as Sloop, we would like to have a high standard on the individuals marked as a match becuase mistakes are extremely detrimental. It can easily propagate over the whole database so we would like to select a classifier with high F-beta score, preferably with high precision. In general, SVM-RBF is a very safe choice. However, the classifier with highest F-beta score are listed as following:

\begin{itemize}
  \item Gr$^{L}$-I SVM-RBF
  \item Ot$^{L}$-I PA-II/SVM/SVM-RBF (Even though PA-II yields highest F-score, its precision is very low.)
  \item Gr$^{L}$-II SVM
  \item Ot$^{L}$-II PCT/PA-I/SVM
\end{itemize}

Species-wise, classification results for Otago skinks seem to have higher recall, but lower precision, than that of Grands despite the similar proportion between the matching and non-matching pairs in training, validation, and test sets. This implies that the feature vectors of Otago are pretty similar to one another in the classifiers' point of view. The similarity results from the visual patterns of the species, which are less detailed compared to Grand.

% Conclusion
% * Describe each problem, research design, and findings (ans to prob)
% * Recommendations
\appendix
Suppose each HIT is distributed to $n$ workers, or in other words, each HIT
contains $n$ assignments and there are $m$ questions for each assignments.  For
each assignment, the chance that an assignment is accepted with an incorrect
answer to the unknown pair is $\frac{1}{2^m}$, and the chance that an assignment
is rejected is $\epsilon_m = (1-\frac{1}{2^{m-1}})$

The probability that only one assignment out of $n$ assignments is accepted with
an incorrect answer to the unknown pair is
$$\epsilon_m^{n-1} \cdot \frac{1}{2^m}$$
The probability that two assignments are accepted, but the resulting answer to
the unknown pair is incorrect is
$$\epsilon_m^{n-2} \cdot \binom{2}{2}\frac{1}{2^{2\cdot m}}$$
if both are incorrect, and
$$\epsilon_m^{n-2} \cdot \binom{2}{1}\frac{1}{2^{2\cdot m}}$$
if both answers are `do not match' and the actual answer is the same.
The error rate in this case is thus
$$3 \cdot \epsilon_m^{n-2} \cdot \frac{1}{2^{2\cdot m}}$$
if both answers are `do not match' and the actual answer is the same.
The probability that three assignments are accepted, but the resulting answer to
the unknown pair is incorrect is
$$\epsilon_m^{n-3} \cdot \binom{3}{3}\frac{1}{2^{3\cdot m}}$$
if all are incorrect, and
$$\epsilon_m^{n-3} \cdot \binom{3}{2}\frac{1}{2^{3\cdot m}}$$
if two are incorrect.
The error rate in this case is thus
$$4 \epsilon_m^{n-3} \cdot \frac{1}{2^{3\cdot m}}$$
if two are incorrect.
Let $i$ be the number of assignments accepted and
$\epsilon_{nm} = \frac{1}{2^m\epsilon_m}$.
The error rate can be generalized as follows:
\begin{description}
  \item[$i$ is odd:]
  $$(\binom{i}{i} + \binom{i}{i-1} + \cdots +
  \binom{i}{\frac{i+1}{2}})\epsilon_{nm}^i \epsilon_m^n$$
  \item[$n$ is even:]
  $$(\binom{i}{i} + \binom{i}{i-1} + \cdots +
  \binom{i}{\frac{i}{2}})\epsilon_{nm}^i \epsilon_m^n$$
\end{description}
By the binomial theorem, the total error rate is bounded by

When the only person whose assignment is correct
\include{appb}
\include{biblio}
\end{document}

